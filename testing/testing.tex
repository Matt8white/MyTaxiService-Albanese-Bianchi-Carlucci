\documentclass[a4paper, 12pt]{article}
\usepackage[T1]{fontenc}
\usepackage[utf8]{inputenc}
\usepackage{graphicx}
\usepackage{hyperref}
\usepackage{tabularx}
\usepackage{url}

\hypersetup{
    pdfborder={0 0 0}
}

\newcounter{tc}

\begin{document}

\newcommand{\code}[1]{
    \texttt{#1}
}

\newcommand{\testx}[7]{
	\stepcounter{tc}
	\subsection{Test \thetc: #1} % (fold)

	% subsubsection test_1 (end)
	\begin{tabular}{l p{0.7\textwidth}}
    \hline
    \textbf{Test Case Identifier} & \thetc\\
    \hline
    \textbf{Test Item(s)} & \code{#2} $\rightarrow$ \code{#3}\\
    \hline
    \textbf{Input Specification} & #4\\
    \hline
    \textbf{Output Specification} & #5\\
    \hline
    \textbf{Environmental Needs} & #6\\
    \hline
    \textbf{Test Description} & #7\\
    \hline
	\end{tabular}
}

\title{Integration Test Plan Document}

\author{M. Albanese, M. Bianchi, A. Carlucci}

\maketitle
\newpage{}
\tableofcontents{}

\newpage{}

\section{Introduction}
% \subsection{Revision History}
% \label{sub:revision_history}

\subsection{Purpose}
\label{sub:purpose}
The purpose of this document, the \textbf{Integration Test Plan Document} (ITPD), is to describe how integration tests are to be performed.

The tests that will be described here will mainly focus on the flow of information between different modules rather than on the modules themselves.
In particular, it describes the adopted methodologies, the sets of all tests to be performed, the tools that will be used during the whole process.

\subsection{Scope} % (fold)
\label{sub:scope}
The system will be an optimization of a pre-existing, non-software solution for renting taxis already in use in the city. The new system will let users to rent or reserve a taxi through a mobile or a web application and will also let taxi drivers to take care of the users' requests in a more simple and effective way. In addition to a better user interface, the new system will focus on a smarter organization of the vehicles deployed in each city zone, resulting in a more efficient service for the citizens.

% subsection scope (end)

\newpage
\subsection{List of definitions and abbreviations}
\label{sub:list_of_definitions_and_abbreviations}

\begin{description}
    \item[JUnit] The tool used for unit testing. See Section~\ref{sub:tools_and_test_equipment_required} for more information.
    \item[Mockito] A mocking framework used in conjuction with JUnit.
    \item[Arquillian] The tool used for the actual integration testing. See Section~\ref{sub:tools_and_test_equipment_required} for more information.
    \item[RASD] Requirements \& Analysis Specification Document
    \item[DD] Design Document
    \item[ITPD] Integration Test Plan Document
    \item[DBMS] Database Management System
    %\item[]
\end{description}

\newpage
\section{Integration Strategy}
\label{sec:integration_strategy}

\subsection{Entry Criteria}
\label{sub:entry_criteria}
Several entry criteria must be met before Integration Testing phase.

The whole architecture must be designed according to the contents of the \textbf{Design Document}, to be written along with this one.
Following this phase, each component that will be integrated must be - of course - coded.

After development phase, each module must successfully pass a thorough \textbf{unit testing}, which guarantees the absence of critical bugs in the codebase. As stated in Section~\ref{sub:tools_and_test_equipment_required}, JUnit will be the privileged tool for this phase.

Along with unit testing, code inspection is done, via both manual and automated work; all these devices will lead to well-formed and robust modules, ready to be composed together.

\subsection{Elements to be integrated}
\label{sub:elements_to_be_integrated}

\begin{figure}[htb]
    \centering
    \includegraphics[width=1.2\textwidth]{img/elements.pdf}
    \label{fig:testplan}
\end{figure}

\begin{table}
    \centering
    \begin{tabular}{| l | l | p{0.35\textwidth} | p{0.3\textwidth} |}
    \hline
    \textbf{ID} & \textbf{Component} & \textbf{Subsystem} \\
    \hline
    A & DAOs & DataLayer \\
    \hline
    1 & GuestManager & ClientMgr\\
    \hline
    2 & Customer & DataLayer\\
    \hline
    3 & CustomerManager & ClientMgr\\
    \hline
    4 & Address & DataLayer\\
    \hline
    5 & Call & DataLayer\\
    \hline
    6 & Reservation & DataLayer\\
    \hline
    7 & Request & DataLayer\\
    \hline
    8 & Ride & DataLayer\\
    \hline
    9 & TaxiResMgrManager & TaxiResMgr\\
    \hline
    10 & RideManager & TaxiResMgr\\
    \hline
    11 & TaxiAllocationDaemon & TaxiAllocationMgr\\
    \hline
    12 & Zone & DataLayer\\
    \hline
    13 & TaxiHandler & TaxiAllocationMgr\\
    \hline
    14 & Taxi & DataLayer\\
    \hline
    15 & TaxiDriverManager & TaxiDriverMgr\\
    \hline
    \end{tabular}
    \caption{List of all components to be integrated. The list of all Data Access Objects is omitted for simplicity.}
    \label{tab:components}
\end{table}

\subsection{Integration Testing Strategy}
\label{sub:integration_testing_strategy}
The strategy that will be adopted is the so-called \textbf{bottom-up} approach.

An incremental approach is fundamental, in order to prevent all shortcomings that are typical to \emph{Big Bang approach} (you have to wait until all modules are complete, localizing the faulty components can be difficult, some interfaces could be missed easily during testing...).

We decided to choose bottom-up over top-down because there are many components at the lower levels (see Figure~\ref{fig:testplan}); this means that only few drivers, if any, are needed.

\newpage
\subsection{Sequence of Component/Function Integration}
\label{sub:sequence_of_component_function_integration}
As the system consists of several different parts interrelated one to the other, we decided to plan integration testing under two different points of view, in accordance to what was written in \cite[p.~4]{bib:dd}.

In particular, Section~\ref{ssub:software_integration_sequence} will explain how the main software components will be integrated, as they were defined in \cite{bib:dd} in the \emph{High Level Components} section, whilst the following section will explain how integration will take part on a upper level by considering subsystems only.

\subsubsection{Software Integration Sequence}
\label{ssub:software_integration_sequence}
\begin{table}
    \centering
    \resizebox{1\textwidth}{!}{
    \begin{tabular}{| l | l | l | l | l |}
    \hline
    \textbf{Test ID} & \textbf{Component 1} & \textbf{Subsystem 1} & \textbf{Component 2} & \textbf{Subsystem 2} \\
    \hline
    A & DAOs & DataLayer & DBMS & DBMS \\
    \hline
    B & EntityBeans & DataLayer & DAOs & DataLayer \\
    \hline
    1 & Taxi & DataLayer & DBMS & DBMS \\
    \hline
    2 & Zone & DataLayer & DBMS & DBMS \\
    \hline
    3 & Ride & DataLayer & DBMS & DBMS \\
    \hline
    4 & Customer & DataLayer & DBMS & DBMS \\
    \hline
    5 & Address & DataLayer & DBMS & DBMS \\
    \hline
    6 & Call & DataLayer & DBMS & DBMS \\
    \hline
    7 & Request & DataLayer & DBMS & DBMS \\
    \hline
    8 & Reservation & DataLayer & DBMS & DBMS \\
    \hline
    9 & TaxiDriverManager & TaxiDriverMgr & Taxi & DataLayer \\
    \hline
    10 & TaxiHandler & TaxiAllocationMgr & Taxi & DataLayer \\
    \hline
    11 & TaxiHandler & TaxiAllocationMgr & Ride & DataLayer \\
    \hline
    12 & RideManager & TaxiReservationMgr & Ride & DataLayer \\
    \hline
    13 & CustomerManager & ClientMgr & Customer & DataLayer \\
    \hline
    14 & GuestManager & ClientMgr & Customer & DataLayer \\
    \hline
    15 & TaxiResManager & TaxiReservationMgr & RideManager & TaxiReservationMgr \\
    \hline
    16 & TaxiAllocationDaemon & TaxiAllocationMgr & TaxiHandler & TaxiAllocationMgr \\
    \hline
    17 & TaxiAllocationDaemon & TaxiAllocationMgr & Zone & DataLayer \\
    \hline
    18 & TaxiAllocationDaemon & TaxiAllocationMgr & Ride & DataLayer \\
    \hline
    19 & TaxiResManager & TaxiReservationMgr & Call & DataLayer \\
    \hline
    \end{tabular}
    }
    \caption{List of all components to be integrated}
    \label{tab:components-integration}
\end{table}
First of all, all transactions with the DB must be operative in order to proceed with all other integrations: for this reason, all modules interacting with the DBMS are tested first.

After that, \code{Client} and \code{TaxiReservation} are tested, so that \code{Client} is completely functional.\footnote{As written in \cite{bib:dd}, \code{Client} contains the data and the logic for managing registered users and guests. \code{TaxiReservation} is needed for the retrieval of all calls done by a specified user}

Since the allocation depends on at least a call, the integration between \code{TaxiAllocation} and \code{TaxiReservation} is done next, followed by the integration between \code{TaxiDriver} (which contains all logic for updating a specified taxi) and \code{TaxiAllocation} (which contains the code for explicitly reserving a taxi).

Eventually the core is complete; the last integration test is done on the mobile app of each taxi driver and the \code{TaxiDriver} component (for example, to check if the \code{updatePosition()} method is working fine).

\begin{table}
    \centering
    \begin{tabular}{| l | l | l |}
        \hline
        \textbf{N.} & \textbf{Subsystem} & \textbf{Integrates with} \\
        \hline
        1 & Client & DBMS \\
        2 & TaxiRes & DBMS \\
        3 & TaxiAllocation & DBMS \\
        4 & Client & TaxiReservation \\
        5 & TaxiAllocation & TaxiReservation \\
        6 & TaxiDriver & TaxiAllocation \\
        7 & Mobile App & TaxiDriver \\
        \hline
    \end{tabular}
    \caption{Integration order of the components}
    \label{tab:component-integration}
\end{table}

\subsubsection{Subsystem Integration Sequence}
\label{ssub:subsystem_integration_sequence}

Four main subsystems make up the whole architecture, listed in Table~\ref{tab:subsystem-integration}. All components described in \cite{bib:dd} make up the Core subsystem, which contains the main logic of the service; as stated in the previous section, the first important integration to be done is the one with the DBMS. When the core system is ready, the mobile application for a taxi driver must be tested in conjunction with the main service. Finally, the UI must be integrated with the business tier. 

\begin{table}
    \centering
    \begin{tabular}{| l | l | l |}
        \hline
        \textbf{N.} & \textbf{Subsystem} & \textbf{Integrates with} \\
        \hline
        1 & Core (Back-end) & DBMS \\
        2 & Taxi Driver mobile application (logic and UI) & Core \\
        3 & Mobile UI & Core \\
        4 & Web UI & Core \\
        \hline
    \end{tabular}
    \caption{Integration order of the subsystems composing the whole architecture}
    \label{tab:subsystem-integration}
\end{table}

\newpage
\section{Individual Steps and Test Description}
\label{sub:individual_steps_and_test_description}

% === DB ===
\testx{Taxi - Access to DB}{Taxi}{DBMS}{Frequent queries on table Taxi}{The corresponding tuples to the query}{DB working}{Verify that all entities in the DB are correctly mapped through JavaBeans}

\testx{Zone - Access to DB}{Zone}{DBMS}{Frequent queries on table Zone}{The corresponding tuples to the query}{DB working}{Verify that all entities in the DB are correctly mapped through JavaBeans}

\testx{Ride - Access to DB}{Ride}{DBMS}{Frequent queries on table Ride}{The corresponding tuples to the query}{DB working}{Verify that all entities in the DB are correctly mapped through JavaBeans}

\testx{Customer - Access to DB}{Customer}{DBMS}{Frequent queries on table Customer}{The corresponding tuples to the query}{DB working}{Verify that all entities in the DB are correctly mapped through JavaBeans}

\testx{Address - Access to DB}{Address}{DBMS}{Frequent queries on table Address}{The corresponding tuples to the query}{DB working}{Verify that all entities in the DB are correctly mapped through JavaBeans}

\testx{Call - Access to DB}{Call}{DBMS}{Frequent queries on table Call}{The corresponding tuples to the query}{DB working}{Verify that all entities in the DB are correctly mapped through JavaBeans}

\testx{Request - Access to DB}{Request}{DBMS}{Frequent queries on table Request}{The corresponding tuples to the query}{DB working}{Verify that all entities in the DB are correctly mapped through JavaBeans}

\testx{Reservation - Access to DB}{Reservation}{DBMS}{Frequent queries on table Reservation}{The corresponding tuples to the query}{DB working}{Verify that all entities in the DB are correctly mapped through JavaBeans}

% === OTHERS ===
\testx{TaxiDriverManager - Update Taxi entity}{TaxiDriverManager}{Taxi}{Update information on position and status of the taxi}{Taxi has modified data}{Working Glassfish Server}{The purpose of this test is to verify that all the information are correctly updated on the server}

\testx{TaxiHandler - Taxi assignment to a ride}{TaxiHandler}{Taxi}{New ride available}{Ride assigned to taxi}{Working Glassfish Server}{Verify that a new ride is properly assigned to a taxi in the correct zone}

\testx{TaxiHandler - Ride completion with a taxi}{TaxiHandler}{Ride}{New ride on schedule}{Taxi assigned to the specified ride}{Working Glassfish Server}{Verify that a taxi is correctly assigned to the ride}

\testx{RideManager - Operations on rides}{RideManager}{Ride}{??}{??}{??}{???}

\testx{CustomerManager - ???}{CustomerManager}{Customer}{??}{??}{??}{???}

\testx{GuestManager - ???}{GuestManager}{Customer}{??}{??}{??}{???}

\testx{TaxiResManager - ???}{TaxiResManager}{RideManager}{??}{??}{??}{???}

\testx{TaxiAllocationDaemon - ???}{TaxiAllocationDaemon}{TaxiHandler}{??}{??}{??}{???}

\testx{TaxiAllocationDaemon - ???}{TaxiAllocationDaemon}{Zone}{??}{??}{??}{???}

\testx{TaxiAllocationDaemon - ???}{TaxiAllocationDaemon}{Ride}{??}{??}{??}{???}

\testx{TaxiResManager - ???}{TaxiResManager}{Call}{??}{??}{??}{???}

\newpage
\section{Tools and Test Equipment Required}
\label{sub:tools_and_test_equipment_required}

\begin{description}
    \item[jUnit:] a simple framework to write repeatable tests. Each test is done on a single unit, usually composed of one public class. Tests can also be grouped in Suites for multiple instances at once.

    Since complex environments must be described within this project (for example, the interaction between \code{TaxiHandler} and \code{Taxi}s or all interactions between a module and the DBMS), a mock framework is necessary. Among several and similar products(JMock, EasyMock, Powermock...), we decided to use \textbf{Mockito} for its simplicity and clearness.

    These tools are used \emph{before} Integration Testing happens (described throughout this document).

    \item[Arquillian] ??????????
\end{description}

\newpage
\section{Program Stubs and Test Data Required}
\label{sub:program_stubs_and_test_data_required}

\appendix

\clearpage
\addcontentsline{toc}{section}{References}

\begin{thebibliography}{9}
\bibitem{bib:assignment}
    Prof. Di Nitto - \emph{Assignment 4 - Integration Test Plan}

\bibitem{bib:spingrid}
    AA.VV. - \emph{SpinGrid - Integration Test Plan Example}

\bibitem{bib:rasd}
        Albanese Michele, Bianchi Mattia, Carlucci Alain - \emph{myTaxiService: Requirements Analysis and Specification Document}

\bibitem{bib:dd}
        Albanese Michele, Bianchi Mattia, Carlucci Alain - \emph{myTaxiService: Design Document}

\bibitem{bib:arquillian}
        AA.VV. - \emph{Arquillian Reference Guide} {\small(\url{https://docs.jboss.org/author/display/ARQ/Reference+Guide?_sscc=t})}

\bibitem{bib:mockito}
        AA.VV. - \emph{Mockito Reference Guide} {\small(\url{http://site.mockito.org/mockito/docs/current/org/mockito/Mockito.html})}
\end{thebibliography}

\vfill

% \section*{Hours spent}
% \begin{figure}[htb]
%     \centering
%     \includegraphics[width=1.2\textwidth]{img/hours.pdf}
%     \label{fig:hours}
% \end{figure}

\end{document}
