\section{Introduction}

\subsection{Purpose}
This is the Requirement Analysis and Specification Document (RASD from now on). The aim of this document is to show the functional and non-functional requirements of the system-to-be, based on several important aspects: the needs expressed by the stakeholders, the constraints which it is subject to, the typical scenarios that will happen after its deployment. The targeted audience is mainly made of software engineers and developers who have to actually develop the service here described.

\subsection{Scope}
The system will be an optimization of a pre-existing, non-software solution for renting taxis already in use in the city. The new system will let users to rent or reserve a taxi through a mobile or a web application and will also let taxi drivers to take care of the users' requests in a more simple and effective way. In addition to a better user inteface, the new system will focus on a smarter organization of the vehicles deployed in each city zone, resulting in a more efficient service for the citizens.

\subsection{Actors}
\begin{itemize}
	\item \textbf{Guest}: A person who has not logged in yet. He can only sign up or sign in if already registered in the system.
	\item \textbf{Customer}: A person who has already registered as a client of the service via mobile or web app. He can call for a taxi, reserve one or modify his own profile info.
	\item \textbf{Taxi driver}: This user is registered as a counterpart of an actual driver. He can notify his availability to the system and accept a request for a ride.
\end{itemize}

% goals:
%  1) garantire l'accesso agli utenti
%  2) garantire un fair management delle code
%  3) permettere al tassista di indicare la propria disponibilita'
%  4) garantire la presenza di taxi in ogni zona
%  5) fare le API
%  6) garantire possibilita' di prenotazione taxi per il momento
%  7) possibilita' riservare taxi a una data/ora con src/dest
%  8) garantire condivisione di un taxi
%  9) tenere aggiornato l'utente sullo stato delle prenotazioni
% 10) garantire piu' possibilita' di pagamento

\subsection{Definitions, acronyms and abbreviations}
\subsection{References}
\subsection{Overview}
