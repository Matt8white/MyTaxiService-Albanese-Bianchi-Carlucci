\pagebreak
\section{Specific requirements}

% Counter Goals
\newcounter{FunReqG}
% Counter Requirements, si resetta quando cambia quello dei goal
\newcounter{FunReqR}[FunReqG]
% Counter Dom. Assumptions, si resetta quando cambia quello dei goal
\newcounter{FunReqD}[FunReqG]

% --- TUTORIAL SU COME FARE GOALS ---
% \begin{Goal}{etichetta_x_refs}{Titolo}{
%     \Req{...}
%     \Dom{...}
% }
% \end{Goal}
% --- END TUTORIAL ---
%

% creo l'environment goal
\newenvironment{Goal}[2]{
    \refstepcounter{FunReqG}
    \paragraph{Goal \arabic{FunReqG}:} #2.
    \label{goal:#1}
    \begin{itemize}
}{\end{itemize}}

\newcommand{\Req}[1]{
    \stepcounter{FunReqR}
    \item[] \textbf{Requirement \arabic{FunReqR}}: #1.
}

\newcommand{\Dom}[1]{
    \stepcounter{FunReqD}
    \item[] \textbf{Domain Assumption \arabic{FunReqD}}: #1.
}

\subsection{Functional Requirements}

\begin{Goal}{register}{Allow guests to register to the platform}{
    %\Dom{The guest must not be already registrated}
    %\Dom{The email used for registration must not be already
%used by another user}
    \Req{The service shall prevent guests from accessing any service before being registered or logged in}
    \Req{The system shall validate any input by the guest}
    \Req{The system shall send a verification link to the user who has just signed up in less than 5 minutes} 	
    \Req{The system shall expire the validation link if not used after 48 hours} 	
}
\end{Goal}

\begin{Goal}{login}{Allow users to access the platform by logging in}{
    \Req{The system shall validate any input by the system before sending it}
    \Req{The system shall check in the DB if user and password are correct and grant access if so}
    \Req{The system shall prevent anyone from logging more than once at a time} 
}
\end{Goal}
	
\begin{Goal}{queue}{Guarantee a fair management of the zone queues}{
    \Req{The system shall send a notification to the taxi driver on top of the queue when a request comes from a customer in the same zone} 
    %\Req{Taxi drivers must positively answer to the notification within 30 seconds}
    \Dom{There is always a taxi driver in every queue}
    \Dom{A driver in a queue is available to accept a ride}
}
\end{Goal}

%????
% \subsubsection{[G3] A driver who misses or refuses a call is removed from the queue and reinserted at the bottom of it.}
% 	\begin{enumerate}
% 	\item \textbf [R1] ??
% 	\end{enumerate}
%????	

\begin{Goal}{availability}{Allow taxi drivers to indicate their availability}{
    \Req{The system shall provide two buttons on the taxi driver's app, "Notify Availability" and "Notify Unavailability"}
    \Req{The system shall let a taxi driver to notify his availability only when unavailable and viceversa}
    \Dom{A taxi driver notifies his availability iff he is actually available for a ride (not busy in another one nor not working at the moment)}
}
\end{Goal}

\begin{Goal}{zone}{Guarantee presence of taxi drivers in every zone}{
   \Req{The system shall assign each taxi to a zone only if the number of taxis in that zone is between a minimum and a maximum threshold, determined according to the expected traffic}
   \Dom{The number of drivers is greater than the number of zones}
}
\end{Goal}
    
\begin{Goal}{api}{Allow developers to add functionalities to the system}{
   \Req{The system must provide APIs in at least a programming language}
   \Req{The system shall provide all basic functionalities that can be found in the web/mobile app through these APIs}
   \Req{The system must check that every command submitted by APIs could be run at that user level of privileges}
}
\end{Goal}
    
\begin{Goal}{call}{Allow customers to request a taxi}{
   \Req{The system shall let the user insert starting point and destination}
   \Req{The system shall check if the input is valid (aka the places actually exist)}
   \Req{The system shall send the request as soon as the customer clicks on the Request button}
  \Req{The system shall show a confirmation screen with the Taxi ID and expected time of arrival}
\Req{The system shall let the user choose a payment method between cash and credit card} 
   \Dom{ If a user requests a taxi, he actually wants to use it }
}
\end{Goal}

\begin{Goal}{allocate}{Allow the system to efficiently allocate a taxi}{
 \Req{When a customer calls for a taxi using his app, the system must deliver a request to the first taxi driver in the customer's zone queue that meets the payment prerequisites (e.g. only POS-enabled taxis must be used when the customer decided to use a credit card as payment method)}
\Req{The system shall set the taxi driver status to Busy when a request is accepted by him}
\Req{The system shall dequeue the taxi driver from the front and enqueue it in the bottom if he does not accept a request}
\Req{The system shall definitely dequeue a taxi driver if he refuses 3 rides in a row}
\Dom{If a taxi driver wants to take care of a request, he accepts it}
}
\end{Goal}

\begin{Goal}{reserve}{Allow customers to reserve a taxi for a given date and time}{
   \Req{The system shall let the user insert starting point and destination}
    \Req{The system shall prevent the customer from reserving a taxi in less than 2 hours}
   \Req{The system shall start the allocation process only 10 minutes before the reservation time}
   \Req{The system shall check if the input is valid (aka the places actually exist)}
   \Req{The system shall let the user choose a payment method between cash and credit card}

   \Dom{The customer will actually show up at the given place, date and time}
}
\end{Goal}

\begin{Goal}{share}{Allow customers the possibility to share a ride}{
   \Req{The system shall provide a Sharing option when reserving a taxi}
   \Req{The system shall automatically insert the customer willing to share in the first shared-reserved taxi if the info provided coincide and at least one taxi meet these requirements}
   \Req{If a user wants to share a ride but no taxis are already shared-reserved, the system shall let the user reserve a new one as usual}
   \Req{The system shall automatically elaborate the shortest path to reach each customer for that ride}
  \Req{The system shall calculate each fee according to the actual distance run by each customer}  
   \Dom{Given a place, the system can always detect the zone where it is}
   \Dom{Customers who decided to share a taxi always accept other customers without any discrimination}
}
\end{Goal}

% -- TUTORIAL REFS+LABEL --
% Il goal \ref{goal:info} si trova a pagina \pageref{goal:info}
% -- END TUTORIAL --

% --- TUTORIAL SCENARIOS --- 
% \scenario{etichetta}{titolo}{testo}
% --- END TUTORIAL ---

\pagebreak
\subsection{Scenarios}
\scenario{michael}{A shy student}{Michael does not want to lose his favorite class, but unfortunately a strike has been announced for today. Since he lives not too far nor too near from his university, he decides to take a taxi; unfortunately, his shyness does not let him call for one. So he decides to use MyTaxiService, the application developed by his own city, in order to get a cab as soon as possible. As a registered user, the only thing he has to do is open the app, log in, click on the “Get a taxi now” button and just wait for the response to come.}

\scenario{siegfried}{The german IT entrepreneur}{Siegfried is a German IT entrepreneur who has just arrived to the city for a working meeting. Since he does not trust the public transport service, he wants to rely on the taxi service. After a quick search on Google, he finds myTaxiService and decides to download the app. After filling in his own personal data, including the cell phone number, he clicks on the Submit button; after few seconds, he receives an email with a link so that he can confirm his registration and start using the service.}
%(SMS??)

\scenario{norma}{Taxi reservation}{Norma has to fly to Denmark the next day early in the morning. Due to this, she cannot take any public transportation service. Being a very organized woman, she decides to reserve a taxi the night before. Since she has very little time, she wants a friendly and fast way to do so; she decides then to use the “Reseve a taxi” functionality provided by myTaxiService web app on her laptop. The only things she has to insert are her starting point, her destination and when to leave. After that, a confirmation message is given and a taxi is successfully reserved.}

\scenario{riccardo}{The amusement park lover}{Riccardo loves amusement parks and so he decides to organize a full day in the nearest one. He knows that the trains and busses will be pretty unusable; since he wants to spend as much as possible on his entertainment, he decides to reserve a taxi the night before by using the Sharing option. Luke and Christine decide to do the same too and find out that Riccardo had already reserved a vehicle for the same date and the same starting point. They accept to take the same taxi and equally divide the fees.}

\scenario{max}{The taxi driver}{Max is a driver working for the public taxi service of his own city. During his daytime turn, he gets a request for a ride from a man nearby; Max accepts since the application has not notified anything at the moment. During the ride a request comes from the system, but Max refuses to accept it by clicking on the “Decline Request” button. The system puts the driver in his zone queue tail}

% --- TUTORIAL USECASE --- 
% \usecase{Name}{Description}{PrimaryActor}{BasicFlow}{AlternateFlow}
% --- END TUTORIAL ---
\pagebreak
\subsection{Use cases and UML diagrams}
INSERT DIAGRAM HERE
\includegraphics[scale=0.5]{wip.png}
\pagebreak

% ---------- SIGN UP --------------------
\usecase
{Sign Up}
{Registration to the system by a customer-to-be.}
{Guest}
{
\begin{enumerate}
	\item Guests visit the service home page
	\item The guest clicks on "Register"
	\item The system outputs the form on screen
	\item The guest inserts his name, mail, password and mobile phone number and then clicks on "Continue"
	\item The system checks if the mail, password and phone number are valid; if so, it sends a mail and a sms with a verification link
	\item The guest clicks on the link and verifies his account
	\item The guest is now a customer registered to the service. 
\end{enumerate}
}
{
/
}
{ 
\begin{enumerate}
	\item The guest is already registered
	\item One or more fields are not well-formed
	\item Username already in use
    \item The validation link is not more valid (after 48 hours)
\end{enumerate}
}

% SEQUENCE 
\pagebreak
\includegraphics[scale=0.5]{wip.png}

% ---------- SIGN IN --------------------
\usecase
{Sign in}
{Login by a customer}
{Guest, Customer}
{
\begin{enumerate}
	\item A guest visits the service home page
	\item The guest clicks on "Log in"
	\item The system outputs the form on screen
	\item The guest inserts his name, password then clicks on "Continue"
	\item The system checks if the mail, password and phone number are valid and present in the DB
	\item The guest is redirected to the home page and recognized as a customer from now on.
\end{enumerate}
}
{
/
}
{ \begin{enumerate}
	\item The user and/or password are incorrect
	\item One or more fields are not well-formed
	\item Username not registered
\end{enumerate} }

% SEQUENCE 
\pagebreak
\includegraphics[scale=0.5]{wip.png}

% ---------- RESERVATION --------------------
\usecase
{Reserve a taxi}
{Reservation of a taxi by a customer for a specified date, start and destination}
{Customer, Taxi Handler}
{
\begin{enumerate}
	\item The customer clicks on "Reserve a taxi"
	\item The system shows the reservation page
	\item The customer fills in the mandatory information (starting point, destination and date-time)
	\item The TaxiHandler registers the request and evaluate it when needed (see "Allocate a taxi" use case)
	\item The customer is informed that the operation did not fail thanks to a confirmation screen
\end{enumerate}
}
{
\textbf{Sharing option}
\begin{enumerate}
	\item Same operations as before till point 3
	\item The user also specifies that he wants to share his taxi
	\item The TaxiHandler assigns the customer to the first reserved taxi for that ride if present
	\item If there are no available shared rides, a new one is created on the spot
	\item Same operation as before from point 4 on
\end{enumerate}
}
{ 
\begin{itemize}
\item The date and/or places are not valid
\end{itemize}
}

% SEQUENCE 
\pagebreak
\includegraphics[scale=0.5]{wip.png}

% ---------- REQUEST --------------------
\usecase
{Request a taxi}
{Quick taxi request for a customer who needs a taxi for the immediate future at its location}
{Customer, Taxi Handler}
{
\begin{enumerate}
	\item The customer clicks on "Request a taxi"
	\item The system take charge of the customer's request by selecting the right "zone queue" according to the user's gps location
	\item The TaxiHandler processes the call (see "Allocate a taxi" use case)
	\item The customer is informed that the operation did not fail thanks to a confirmation screen
\end{enumerate}
}
{
/
}
{ 
\begin{itemize}
\item No taxi available in the specified zone queue
\end{itemize}
}

% SEQUENCE 
\pagebreak
\includegraphics[scale=0.5]{wip.png}

% ---------- ALLOCATE --------------------
\usecase
{Allocate a taxi}
{Selection and notification of a taxi driver to process a request or a reservation by the TaxiHandler}
{Taxi Handler, Taxi Driver}
{
\begin{enumerate}
	\item The TaxiHandler sends a notification to the first taxi driver in the given zone queue
	\item If a positive answer is given in 30 seconds the TaxiHandler provides to set the Driver as "busy"
	\item Otherwise the driver's "missed calls" counter is increased and the driver is moved to the bottom of the queue. TaxiHandler restarts the seek from step 1
\end{enumerate}
}
{
/
}
{ 
/
}

% SEQUENCE 
\pagebreak
\includegraphics[scale=0.5]{wip.png}

% WORKFLOW 
\pagebreak
\includegraphics[scale=0.5]{wip.png}