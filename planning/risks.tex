%!TEX root = planning.tex
\section{Risks}
Risks have to considered in a complete project planning, owing their uncertain ad dangerous nature. A sudden change in mind, actions, economical situations and alike could drift the project development into failure; this is the reason they are here analyzed. Three main risk categories will be later described:
\begin{itemize}
    \item \textbf{Project risks}: involving the \emph{project plan} (described in these pages). Project schedule and overall costs may be subject to (worse) changes due to these risks.
    \item \textbf{Technical risks}: involving the actual \emph{implementation} of the project. They may affect the quality of the software being developed.
    \item \textbf{Business risks}: involving the \emph{company} developing the software. This may cause trouble to the project (e.g. if the business cannot subsidize the software being developed anymore).
\end{itemize}

\subsection{Project Risks}

\begin{itemize}
    \risk{No estimations/schedules have been made before this project. A lack of experience in this area can lead to serious errors in evaluating development time}{High}{Critical}{Studying previous works on a similar subject can be very helpful in this.}
    \risk{Due to several overlapping tasks the team is involved into, the project is very likely to suffer from schedule delays}{High}{Critical}{A strict organization among the team components is fundamental. This implies a constant cooperation between developers, in order to squeeze even the tiniest time slots available for this project.}
    \risk{A sudden growth in requirements can lead to a rush to meeting deadlines, jeopardizing the overall quality}{Medium}{Critical}{Thinking with a broader mind on the first stages can be very helpful; however, the team should be careful against over-engineering (which can also paralyze the development)}
    \risk{Collaboration issues can sometimes be crucial, especially when dealing with task divisions.}{Medium}{Medium}{Meeting often can be a solution, other than explicitly writing who has to do what}
    \risk{The team is very small (3 people) but homogeneous; if someone leaves or gets ill then the remaining team will have serious repercussions.}{Low}{Catastrophic}{All team members must be able to cover all development sections and cooperate effectively.}
\end{itemize}

\subsection{Technical Risks}

\begin{itemize}
  \risk{A lack of previous experience in developing with Java EE can almost surely slow down the entire team, which has to study these new technologies first}{High}{Critical}{This has to be account in the first stages of planning and inserted in the project scheduling.}
  \risk{If the servers happen to be unreliable or in the case of more users than expected, a significant downtime can seriously damage the whole project}{Medium}{Critical}{A scalable design of the overall architecture is essential, both in software and in hardware choices.}

  \risk{The application may be susceptible to security issues if not well-designed.}{Medium}{Critical}{All modern standards in computer security guidelines must be followed, especially when dealing with the user input, which has to be correctly verified and processed.}

  \risk{Testing may difficult (for example, if several mocks are necessary) or highlight problems which are hard to solve, especially when doing integration testing or ---even worse---validation.}{Medium}{Critical}{All components must be unit tested as soon as possible, to eliminate serious bugs when they first appear; integration testing must be done according to the specifications contained in \cite{bib:itpd}. A periodic check of requirements contained in \cite{bib:rasd} is also required.}

\end{itemize}


\subsection{Business Risks}

\begin{itemize}
    \risk{Testing devices \& infrastructure (PCs, several mobile phones, server rent) need to be purchased and configured. This is going to increase costs, that may be not sustainable if the company is too small.}{High}{Catastrophic}{Testing tools are to be clearly defined in \cite{bib:rasd}, in order to avoid worthless spendings.}

    \risk{The company may find itself in serious financial trouble.}{Low}{Catastrophic}{A feasibility study together with the RASD must highlight the impossibility of starting a whole new project.}
    
    \risk{myTaxiService may violate some (future) laws related to taxi management.}{Low}{Critical}{A periodic check must be done in order to avoid legal consequences. In the case of drastic changes, the whole team must work in order to adapt to the new regulations as soon as possible.}
    
    \risk{Another bigger company could acquire this company.}{Low}{Marginal}{No preventive solutions are available. Note that this is not necessarily a negative thing.}
\end{itemize}