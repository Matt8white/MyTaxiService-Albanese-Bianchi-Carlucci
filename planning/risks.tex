%!TEX root = planning.tex
\section{Risks}
Risks have to considered in a complete project planning, owing their uncertain ad dangerous nature. A sudden change in mind, actions, economical situations and alike could drift the project development into failure; this is the reason they are here analyzed. Three main risk categories will be later described:
\begin{itemize}
    \item \textbf{Project risks}: involving the \emph{project plan} (described in these pages). Project schedule and overall costs may be subject to (worse) changes due to these risks.
    \item \textbf{Technical risks}: involving the actual \emph{implementation} of the project. They may affect the quality of the software being developed.
    \item \textbf{Business risks}: involving the \emph{company} developing the software. This may cause trouble to the project (e.g. if the business cannot subsidize the software being developed anymore).
\end{itemize}

\subsection{Project Risks}

\begin{table}
    \centering
\begin{tabular}[h!tbp]{p{0.485\textwidth} p{0.2\textwidth} p{0.2\textwidth}}
    \hline
    \textbf{Risk} & \textbf{Probability} & \textbf{Damage}\\
    \riskrow{No estimations/schedules have been made before this project. A lack of experience in this area can lead to serious errors in evaluating development time}{High}{Critical}
    \riskrow{Due to several overlapping tasks the team is involved into, the project is very likely to suffer from schedule delays}{High}{Critical}
    \riskrow{A sudden growth in requirements can lead to a rush to meeting deadlines, jeopardizing the overall quality}{Medium}{Critical}
    \riskrow{Communication issues can sometimes be crucial, especially when dealing with task divisions.}{Medium}{Medium}
    
    %\riskrow{Sudden growth in requirements}{x}{x?}

    \hline
\end{tabular}
\end{table}

\subsection{Technical Risks}
\subsection{Business Risks}
\subsection{Recovery Actions}
\subsection{Summary}