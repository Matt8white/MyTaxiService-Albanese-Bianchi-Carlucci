%!TEX root = planning.tex
\section{COCOMO: effort \& cost estimation}
\subsection{Overview} % (fold)
\label{sub:cocomo_overview}
The COCOMO II Cost Estimation Model is a complex estimation technique used 
by thousands of software engineers. It's used to estimate the effort cost
The core of COCOMO II is the use of the Effort Equation to estimate the number
of Person/Month required to develop a project.

We have used this file as reference: \\ 
\url{http://csse.usc.edu/csse/research/COCOMOII/cocomo2000.0/CII_modelman2000.0.pdf}

\subsection{Scale Drivers} % (fold)
\label{sub:scale_drivers}
% TODO tabella scale drivers

\paragraph{PREC} Precedentedness. \\ 
This driver reflects the previous experience that the developers have in this 
field. Actually, this is our first experience, so we think the best value for
our team is \emph{LOW}.

\paragraph{FLEX} Development flexibility. \\
This driver will change due to our flexibility degree in the development.
Our schedule is quite strict, so we choose \emph{LOW} for this project.

\paragraph{RESL} Risk resolution. \\
% TODO Supercazzolare anche qua
It reflects the extension of the risk analysis. We choose 'generally'
because reasons.

\paragraph{TEAM} Team cohesion. \\
This value is correlated to how well the development team know each other. 
In this case we are a very cooperative team, so \emph{VERY HIGH} value is
our choice.

\paragraph{PMAT} Process maturity. \\
TODO ANCHE QUA, BOH

\subsection{Cost Drivers} % (fold)
\label{sub:cost_drivers}

% subsection cost_drivers (end)
% subsection scale_drivers (end)
\subsection{Effort Equation} % (fold)
\label{sub:effort_equation}
