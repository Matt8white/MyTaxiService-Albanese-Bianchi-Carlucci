\pagebreak
\section{Architectural Design}
 
\subsection{Overview}
Starting from a very general depiction of the overall system, the document explains in detail each component, through several class and component diagrams. In addition to this, a deployment view representing the physical implementation of myTaxiService and a description of the main patterns and styles used are provided. Please refer to each subsection for more details.

\subsection{High level components and their interaction}
The whole architecture can be described from two different points of view: 
\begin{itemize}
	\item \textbf{Tier division}: description of each tier, as given in section \ref{sec:deploy}.
	\item \textbf{Functionality}: each component represents a logical collection of correlated classes in the system (either representing data or business rules). NUMERO MANCANTE main components make up myTaxiService:
	\begin{itemize}
		\item \textbf{Client}: here are grouped all classes which refer to a registered customer\footnote{This means that taxi drivers are not represented here}. In particular, a class reflecting the Client table in the DB and a ClientManager for basic operations (login, signup) are provided.
		% TODO: IMAGE MISSING
		\newpage
		\item \textbf{Taxi Res}: here are collected all classes relative to a request or a reservation into the system. Thus, a model for a generic Call is provided along with a TaxiResManager for I/O operations in the DB.
		% TODO: IMAGE MISSING
		\newpage
		\item \textbf{Allocation}: here are represented the classes relative to allocating a taxi after a call has been previously made. Info about zones, rides and taxi drivers are given along with classes that physically manage the queue-handling process and notifications to drivers and customers.
		% TODO: IMAGE MISSING
		\newpage
		%\item ???
	\end{itemize}
	
\end{itemize}

\subsection{Component view}


\subsection{Deployment view}
\label{sec:deploy}
% MITCH
% tutta la menata sui java beans, richieste SOAP...

\subsection{Runtime view}
\subsection{Selected architectural styles and patterns}
This section describes high level patterns we decided to use in myTaxiService.

\paragraph{Deployment} A four-tier architecture is used in this project.
The whole infrastructure is divided in the following tiers:

\begin{itemize}
    \item{\textbf{Client tier} which is composed of web browsers and mobile app.}
    \item{\textbf{Presentation tier} whose aim is to generate user interface and send it to clients.}
    \item{\textbf{Logic tier} which coordinates the application: performs calculation, makes logical decisions and moves data between Presentation tier and Data tier.}
    \item{\textbf{Data tier} mainly made of entity beans and database. 
    Here information is stored and retrieved.}
\end{itemize}

We choose to use this kind of architecture because it provides a model by which
we can create a flexible and scalable system.

\paragraph{Communication} We choose to use a Service-oriented Architecture, 
an architecture which provides application functionality as a set of services, 
and applications that use those software services. 

Services are \textit{loosely coupled} units of functionality that are self-contained, 
Each service is an implementation of some interfaces, which provide a communication schema with
each application. Also, interfaces can be published and invoked.

\paragraph{Structure} Here we choose to use the Object-Oriented Architectural Style.
This is a design paradigm based on the division of responsibilities for a complex system
into small and reusable parts called ``Objects''.
They communicate with each other through interfaces, by sending and receiving messages
or by calling methods in other objects.

The main benefits of this approach are that it is:

\begin{itemize}
    \item{\textbf{Extensible} because there are a lot of structures which ensures that a change of implementation does not imply a change of interface.}
    \item{\textbf{Reusable} if each object should be developed as a reusable small piece of code.}
    \item{\textbf{Testable} because of the incapsulation, which improves testability.}
    \item{\textbf{Understandable} because it maps the application more closely to the real world.}
\end{itemize}

\subsection{Other design decisions}

There are two aspects of myTaxiService we haven't discussed yet: communication 
with smartphones and geolocalization.

\paragraph {Mobile communication}
First, how can we send communication between application server and smartphones.
There is a standard way to do this, which is the one we choose, \textit{Push Notification}.
We decided to support both iOS and Android smartphones. In order to send push
notifications to iOS smartphone, we have to use \textit{Apple Push Notification Service}.
The same applies to Android world, Google has its own push service, called 
\textit{Google Cloud Messaging}.

\paragraph {Geolocalization}
We decided to use Google Maps as geolocalization service because it is highly
reliable, fast and extremely used in the world.
Also, it offers powerful APIs which allow us to use it for a wide variety of use cases.
