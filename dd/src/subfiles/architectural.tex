\pagebreak
\section{Architectural Design}
 
\subsection{Overview}
Starting from a very general depiction of the overall system, the document explains in detail each component, through several class and component diagrams. In addition to this, a deployment view representing the physical implementation of myTaxiService and a description of the main patterns and styles used are provided. Please refer to each subsection for more details.

\subsection{High level components and their interaction}
The whole architecture can be described from two different points of view: 
\begin{itemize}
	\item \textbf{Tier division}: description of each tier, as given in section \ref{sec:deploy}.
	\item \textbf{Functionality}: each component represents a logical collection of correlated classes in the system (either representing data or business rules). NUMERO MANCANTE main components make up myTaxiService:
	\begin{itemize}
		\item \textbf{Client}: here are grouped all classes which refer to a registered customer\footnote{This means that taxi drivers are not represented here}. In particular, a class reflecting the Client table in the DB and a ClientManager for basic operations (login, signup) are provided.
		% TODO: IMAGE MISSING
		\newpage
		\item \textbf{Taxi Res}: here are collected all classes relative to a request or a reservation into the system. Thus, a model for a generic Call is provided along with a TaxiResManager for I/O operations in the DB.
		% TODO: IMAGE MISSING
		\newpage
		\item \textbf{Allocation}: here are represented the classes relative to allocating a taxi after a call has been previously made. Info about zones, rides and taxi drivers are given along with classes that physically manage the queue-handling process and notifications to drivers and customers.
		% TODO: IMAGE MISSING
		\newpage
		%\item ???
	\end{itemize}
	
\end{itemize}

\subsection{Component view}


\subsection{Deployment view}
\label{sec:deploy}
% MITCH
% tutta la menata sui java beans, richieste SOAP...

\subsection{Runtime view}
\subsection{Selected architectural styles and patterns}
This section describes high level patterns we decided to use in this application.

\paragraph{Deployment} A four-tier architecture is used in this project.
The whole infrastructure is divided in the following levels:
\begin{enumerate}
\item{Client tier, which is composed of web browsers and smartphones app}
\item{Presentation tier, whose aim is to bring application server serivces to web browsers}
\item{Business services, which is composed of session beans }
\item{Data-Access tier, which is composed of entity beans and database}
\end{enumerate}

\paragraph{Communication} We choose to use a Service-oriented Architecture, 
an architecture which provides application functionality as a set of services, 
and applications that use those software services. 

Services are engineered as \textbf{loosely coupled}, which means that each service
is an implementation of some interfaces, which provide a communication schema with
an application.
Also, interfaces can be published and invoked.

\paragraph{Structure} Component based? Object oriented? TODO

\subsection{Other design decisions}
