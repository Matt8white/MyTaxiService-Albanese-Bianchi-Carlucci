\section{Introduction}

\subsection{Purpose}
This document shows the architecture underlying myTaxiService, starting from the specifications and requirements described in the RASD document. It is addressed to developers and maintainers primarly. The main focus is to show the major choices about software and hardware, through several UML and UX diagrams in increasing detail: the goal is to explain how each component interact with each other using a standard language.

\subsection{Scope}
The system will be an optimization of a pre-existing, non-software solution for renting taxis already in use in the city. The new system will let users to rent or reserve a taxi through a mobile or a web application and will also let taxi drivers to take care of the users' requests in a more simple and effective way. In addition to a better user interface, the new system will focus on a smarter organization of the vehicles deployed in each city zone, resulting in a more efficient service for the citizens.

\subsection{Definitions, acronyms and abbreviations}
The following are used throughout the document:
\begin{itemize}
\item\textbf{RASD:} Requirements Analysis and Specification Document.
\item\textbf{DD:} Design Document.
\item\textbf{RDBMS:} Relational Data Base Management System.
\item\textbf{DB:} DataBase, handled by a RDBMS.
\item\textbf{JDBC:} Java DataBase Connectivity.
\item\textbf{UI:} User Interface.
\item\textbf{Application server:} the component which provides the application logic and interacts with the DB and with the front-ends.
\item\textbf{Back-end:} term used to identify the Application server
\item\textbf{MVC:} Model-View-Controller.
\item\textbf{SOAP:} Simple Object Access Protocol.
\item\textbf{JAX-WS:} Java API for XML Web Services.
\item\textbf{JPA:} Java Persistence API.
\end{itemize}